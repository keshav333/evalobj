\documentclass[addpoints,11pt]{exam}
\usepackage[latin1]{inputenc}
\usepackage{amsmath}
\usepackage{amsfonts}
\usepackage{amssymb}
\usepackage{graphicx}
\usepackage{algorithm}
\usepackage{algpseudocode}
\usepackage{hyperref}
\usepackage{listings}
\usepackage{xcolor}
\usepackage{tikz}
\usetikzlibrary{positioning,shapes,arrows,shadows}
\usetikzlibrary{automata,positioning,shapes,arrows,backgrounds,fit}

\author{}
\title{\includegraphics[height=0.53in]{/home/sujit/IIITB/misc/iiitbLogo.jpeg} \\ Programming I (Python) \\ Assignment 1}

\date{}
\definecolor{lightblue}{rgb}{0.8,0.93,1.0} % color values Red, Green, Blue
\definecolor{Blue}{rgb}{0,0,1.0} % color values Red, Green, Blue
\definecolor{Red}{rgb}{1,0,0} % color values Red, Green, Blue
\definecolor{Pink}{rgb}{0.7,0,0.2}
\definecolor{links}{HTML}{2A1B81}
\definecolor{mydarkgreen}{HTML}{126215}

\newcommand{\myprod}[0]{\hspace{0.5cm}$\rightarrow$\hspace{0.5cm}}
\newcommand{\mychoice}[0]{\hspace{0.75cm}$|$\hspace{0.25cm}}
\newcommand{\myderiv}[0]{\hspace{0.5cm}$\Rightarrow$\hspace{0.5cm}}

\begin{document}
  \tikzstyle{fun} = [draw=gray, thick, fill=white, rounded corners, rectangle, inner sep=1cm]
\maketitle

\lstset{
	language = Python,
	basicstyle = \small,
	stringstyle = \ttfamily,
	keywordstyle=\color{black}\bfseries,
	identifierstyle=\color{black},
	frameround=tttt,
	numbers=none,
	showstringspaces=false
}
%\pointsinrightmargin

\thispagestyle{head}
\begin{questions}
\question Which of the following are true about Lisp programming language:
\begin{enumerate}
\item Strictly typed
\item Statically typed
\item Dynamically typed
\item Safe
\end{enumerate}\question What are the features of a first class object in a programming language?
\begin{enumerate}
\item Can be called as a procedure
\item Can be passed as a parameter to a function
\item Can be used as a type
\item Can be returned from a function as a value
\item Can be stored in a data-structure
\item Can be imported as a module
\end{enumerate}\question Which of the following are true about OCaml commands:
\begin{enumerate}
\item They are pure commands.
\item They always produce a value.
\item They may produce a value.
\item They do not produce a value.
\end{enumerate}\question Which of the below are true about OCaml type system?
\begin{enumerate}
  \item Statically typed
  \item Dynamically typed
  \item Implicitly typed
  \item Explicitly typed
  \item Both implicitly and explicitly typed
\end{enumerate} 
\question Which of the following statements is true about OCaml expressions:
\begin{enumerate}
\item OCaml expressions can't have side-effects.
\item OCaml expressions evaluate to a single value.
\item Every expression has exactly one type.
\item The type of the expression depends of the values evaluated so far.
\end{enumerate}\question The name of the OCaml debugger is:
\begin{enumerate}
\item ogdb
\item odebug
\item ocamldebug
\item ocamlgdb
\end{enumerate}\question Which of the following are valid variable names in OCaml:
\begin{enumerate}
\item \lstinline@abc@
\item \lstinline@ab_c@
\item \lstinline@Abc@
\item \lstinline@Ab_c@
\item \lstinline@ab-c@
\item \lstinline@ab1@
\end{enumerate}\question When we use one of the OCaml compilers to compile an OCaml program \lstinline@program.ml@, the compiler name and object code file name are related as follows:
\begin{enumerate}
\item ocamlc $\mapsto$ \lstinline@program.mlo@
\item ocamlopt $\mapsto$ \lstinline@program.mlo@
\item ocamlc $\mapsto$ \lstinline@program.mlc@
\item ocamlc $\mapsto$ \lstinline@program.cmo@
\item ocamlc $\mapsto$ \lstinline@program.cmx@
\item ocamlopt $\mapsto$ \lstinline@program.cmo@
\item ocamlopt $\mapsto$ \lstinline@program.cmx@
\end{enumerate}\question In OCaml, When an expression is evaluated, which of following things may happen:
\begin{enumerate}
\item It may evaluate to a value of the same type as the expression.
\item If typechecked successfully, it will never raise an exception.
\item It may not terminate.
\item It is guaranteedd to terminate.
\end{enumerate}
\question Which of the following are true about OCaml programs:
\begin{enumerate}
\item OCaml programs must always be written on the top loop.
\item OCaml programs must be written in a file like in C.
\item OCaml programs must always be compiled explicitly before being executed.
\item OCaml compiler always produces native machine code as output as in C.
\item OCaml compiler always produces bytecode as output as in Java.
\item OCaml compiler can be used to produce either machine code or byte code as per user preference.
\end{enumerate}\end{questions}
\end{document}
